% Options for packages loaded elsewhere
\PassOptionsToPackage{unicode}{hyperref}
\PassOptionsToPackage{hyphens}{url}
%
\documentclass[
]{article}
\usepackage{amsmath,amssymb}
\usepackage{iftex}
\ifPDFTeX
  \usepackage[T1]{fontenc}
  \usepackage[utf8]{inputenc}
  \usepackage{textcomp} % provide euro and other symbols
\else % if luatex or xetex
  \usepackage{unicode-math} % this also loads fontspec
  \defaultfontfeatures{Scale=MatchLowercase}
  \defaultfontfeatures[\rmfamily]{Ligatures=TeX,Scale=1}
\fi
\usepackage{lmodern}
\ifPDFTeX\else
  % xetex/luatex font selection
\fi
% Use upquote if available, for straight quotes in verbatim environments
\IfFileExists{upquote.sty}{\usepackage{upquote}}{}
\IfFileExists{microtype.sty}{% use microtype if available
  \usepackage[]{microtype}
  \UseMicrotypeSet[protrusion]{basicmath} % disable protrusion for tt fonts
}{}
\makeatletter
\@ifundefined{KOMAClassName}{% if non-KOMA class
  \IfFileExists{parskip.sty}{%
    \usepackage{parskip}
  }{% else
    \setlength{\parindent}{0pt}
    \setlength{\parskip}{6pt plus 2pt minus 1pt}}
}{% if KOMA class
  \KOMAoptions{parskip=half}}
\makeatother
\usepackage{xcolor}
\usepackage[margin=1in]{geometry}
\usepackage{color}
\usepackage{fancyvrb}
\newcommand{\VerbBar}{|}
\newcommand{\VERB}{\Verb[commandchars=\\\{\}]}
\DefineVerbatimEnvironment{Highlighting}{Verbatim}{commandchars=\\\{\}}
% Add ',fontsize=\small' for more characters per line
\usepackage{framed}
\definecolor{shadecolor}{RGB}{248,248,248}
\newenvironment{Shaded}{\begin{snugshade}}{\end{snugshade}}
\newcommand{\AlertTok}[1]{\textcolor[rgb]{0.94,0.16,0.16}{#1}}
\newcommand{\AnnotationTok}[1]{\textcolor[rgb]{0.56,0.35,0.01}{\textbf{\textit{#1}}}}
\newcommand{\AttributeTok}[1]{\textcolor[rgb]{0.13,0.29,0.53}{#1}}
\newcommand{\BaseNTok}[1]{\textcolor[rgb]{0.00,0.00,0.81}{#1}}
\newcommand{\BuiltInTok}[1]{#1}
\newcommand{\CharTok}[1]{\textcolor[rgb]{0.31,0.60,0.02}{#1}}
\newcommand{\CommentTok}[1]{\textcolor[rgb]{0.56,0.35,0.01}{\textit{#1}}}
\newcommand{\CommentVarTok}[1]{\textcolor[rgb]{0.56,0.35,0.01}{\textbf{\textit{#1}}}}
\newcommand{\ConstantTok}[1]{\textcolor[rgb]{0.56,0.35,0.01}{#1}}
\newcommand{\ControlFlowTok}[1]{\textcolor[rgb]{0.13,0.29,0.53}{\textbf{#1}}}
\newcommand{\DataTypeTok}[1]{\textcolor[rgb]{0.13,0.29,0.53}{#1}}
\newcommand{\DecValTok}[1]{\textcolor[rgb]{0.00,0.00,0.81}{#1}}
\newcommand{\DocumentationTok}[1]{\textcolor[rgb]{0.56,0.35,0.01}{\textbf{\textit{#1}}}}
\newcommand{\ErrorTok}[1]{\textcolor[rgb]{0.64,0.00,0.00}{\textbf{#1}}}
\newcommand{\ExtensionTok}[1]{#1}
\newcommand{\FloatTok}[1]{\textcolor[rgb]{0.00,0.00,0.81}{#1}}
\newcommand{\FunctionTok}[1]{\textcolor[rgb]{0.13,0.29,0.53}{\textbf{#1}}}
\newcommand{\ImportTok}[1]{#1}
\newcommand{\InformationTok}[1]{\textcolor[rgb]{0.56,0.35,0.01}{\textbf{\textit{#1}}}}
\newcommand{\KeywordTok}[1]{\textcolor[rgb]{0.13,0.29,0.53}{\textbf{#1}}}
\newcommand{\NormalTok}[1]{#1}
\newcommand{\OperatorTok}[1]{\textcolor[rgb]{0.81,0.36,0.00}{\textbf{#1}}}
\newcommand{\OtherTok}[1]{\textcolor[rgb]{0.56,0.35,0.01}{#1}}
\newcommand{\PreprocessorTok}[1]{\textcolor[rgb]{0.56,0.35,0.01}{\textit{#1}}}
\newcommand{\RegionMarkerTok}[1]{#1}
\newcommand{\SpecialCharTok}[1]{\textcolor[rgb]{0.81,0.36,0.00}{\textbf{#1}}}
\newcommand{\SpecialStringTok}[1]{\textcolor[rgb]{0.31,0.60,0.02}{#1}}
\newcommand{\StringTok}[1]{\textcolor[rgb]{0.31,0.60,0.02}{#1}}
\newcommand{\VariableTok}[1]{\textcolor[rgb]{0.00,0.00,0.00}{#1}}
\newcommand{\VerbatimStringTok}[1]{\textcolor[rgb]{0.31,0.60,0.02}{#1}}
\newcommand{\WarningTok}[1]{\textcolor[rgb]{0.56,0.35,0.01}{\textbf{\textit{#1}}}}
\usepackage{graphicx}
\makeatletter
\def\maxwidth{\ifdim\Gin@nat@width>\linewidth\linewidth\else\Gin@nat@width\fi}
\def\maxheight{\ifdim\Gin@nat@height>\textheight\textheight\else\Gin@nat@height\fi}
\makeatother
% Scale images if necessary, so that they will not overflow the page
% margins by default, and it is still possible to overwrite the defaults
% using explicit options in \includegraphics[width, height, ...]{}
\setkeys{Gin}{width=\maxwidth,height=\maxheight,keepaspectratio}
% Set default figure placement to htbp
\makeatletter
\def\fps@figure{htbp}
\makeatother
\setlength{\emergencystretch}{3em} % prevent overfull lines
\providecommand{\tightlist}{%
  \setlength{\itemsep}{0pt}\setlength{\parskip}{0pt}}
\setcounter{secnumdepth}{-\maxdimen} % remove section numbering
\ifLuaTeX
  \usepackage{selnolig}  % disable illegal ligatures
\fi
\usepackage{bookmark}
\IfFileExists{xurl.sty}{\usepackage{xurl}}{} % add URL line breaks if available
\urlstyle{same}
\hypersetup{
  pdftitle={TimeSeriesMarkdown},
  pdfauthor={Paul Hanson, pchanson@wisc.edu, adapted by: Veronica Gibson},
  hidelinks,
  pdfcreator={LaTeX via pandoc}}

\title{TimeSeriesMarkdown}
\author{Paul Hanson,
\href{mailto:pchanson@wisc.edu}{\nolinkurl{pchanson@wisc.edu}}, adapted
by: Veronica Gibson}
\date{2025-08-01}

\begin{document}
\maketitle

\subsection{Time Series Analysis Introduction: Seasonal
Decomposition}\label{time-series-analysis-introduction-seasonal-decomposition}

Project: NSC SWMP Synthesis Catalyst Class

Funding: NERRS Science Collaborative

Author(s): Paul Hanson,
\href{mailto:pchanson@wisc.edu}{\nolinkurl{pchanson@wisc.edu}}

University of Wisconsin-Madison, Center for Limnology

Adapted to R markdown by Veronica Gibson, University of Hawaiʻi at
Mānoa, HIMB

\begin{Shaded}
\begin{Highlighting}[]
\FunctionTok{summary}\NormalTok{(cars)}
\end{Highlighting}
\end{Shaded}

\begin{verbatim}
##      speed           dist       
##  Min.   : 4.0   Min.   :  2.00  
##  1st Qu.:12.0   1st Qu.: 26.00  
##  Median :15.0   Median : 36.00  
##  Mean   :15.4   Mean   : 42.98  
##  3rd Qu.:19.0   3rd Qu.: 56.00  
##  Max.   :25.0   Max.   :120.00
\end{verbatim}

Data originated from NERRS CDMO: \url{https://cdmo.baruch.sc.edu/} and
data was compiled into csvs for all Reserves using code from:

\url{https://github.com/Lake-Superior-Reserve/WQ_SWMP_Synthesis/tree/main/R/Data_processing}
This repository is currently private until analyses are complete, but
will be made public after publication.

This code uses data from Apalachicola Reserve. This folder has been
uploaded to the repo. Code showing how data was split out for Reserves
is in `Data Processing'. This code selects a single variable from a
single reserve site, based on user input, and separates the trend (i.e.,
long-term variability), seasonal component (i.e., annual cycle) and
random component of the signal (variable). Note that the seasonal
component is the calculated ``average seasonal'' signal, so each year in
a multi-year data set will have the same seasonal component. All the
variation is in the trend and random components. After decomposition,
each of the three components are run through the ACF to demonstrate
\#the degree of autcorrelation in the components.

\subsubsection{Data set up}\label{data-set-up}

\begin{Shaded}
\begin{Highlighting}[]
\CommentTok{\# Load the data into separate data frames}
\NormalTok{datMet}\OtherTok{\textless{}{-}}\FunctionTok{read.csv}\NormalTok{(}\StringTok{"APA/met\_apa.csv"}\NormalTok{)}
\NormalTok{datNut}\OtherTok{\textless{}{-}}\FunctionTok{read.csv}\NormalTok{(}\StringTok{"APA/nut\_apa.csv"}\NormalTok{)}
\NormalTok{datWQ}\OtherTok{\textless{}{-}}\FunctionTok{read.csv}\NormalTok{(}\StringTok{"APA/wq\_apa.csv"}\NormalTok{)}

\CommentTok{\# Create a handy year fraction variable for each data frame for plotting}
\NormalTok{datMet}\SpecialCharTok{$}\NormalTok{YearFrac }\OtherTok{=}\NormalTok{ datMet}\SpecialCharTok{$}\NormalTok{year }\SpecialCharTok{+}\NormalTok{ datMet}\SpecialCharTok{$}\NormalTok{month}\SpecialCharTok{/}\DecValTok{12}
\NormalTok{datNut}\SpecialCharTok{$}\NormalTok{YearFrac }\OtherTok{=}\NormalTok{ datNut}\SpecialCharTok{$}\NormalTok{year }\SpecialCharTok{+}\NormalTok{ datNut}\SpecialCharTok{$}\NormalTok{month}\SpecialCharTok{/}\DecValTok{12}
\NormalTok{datWQ}\SpecialCharTok{$}\NormalTok{YearFrac  }\OtherTok{=}\NormalTok{ datWQ}\SpecialCharTok{$}\NormalTok{year }\SpecialCharTok{+}\NormalTok{ datWQ}\SpecialCharTok{$}\NormalTok{month}\SpecialCharTok{/}\DecValTok{12}

\CommentTok{\# Each reserve can have multiple sampling stations}
\CommentTok{\# Determine the number of stations for each data frame}
\NormalTok{uMet }\OtherTok{=} \FunctionTok{unique}\NormalTok{(datMet}\SpecialCharTok{$}\NormalTok{station)}
\NormalTok{uNut }\OtherTok{=} \FunctionTok{unique}\NormalTok{(datNut}\SpecialCharTok{$}\NormalTok{station)}
\NormalTok{uWQ  }\OtherTok{=} \FunctionTok{unique}\NormalTok{(datWQ}\SpecialCharTok{$}\NormalTok{station)}

\CommentTok{\# Print out the unique stations for each data frame}
\FunctionTok{cat}\NormalTok{(}\StringTok{\textquotesingle{}Unique met stations: \textquotesingle{}}\NormalTok{,uMet,}\StringTok{\textquotesingle{}}\SpecialCharTok{\textbackslash{}n}\StringTok{\textquotesingle{}}\NormalTok{)}
\end{Highlighting}
\end{Shaded}

\begin{verbatim}
## Unique met stations:  apaebmet
\end{verbatim}

\begin{Shaded}
\begin{Highlighting}[]
\FunctionTok{cat}\NormalTok{(}\StringTok{\textquotesingle{}Unique nut stations: \textquotesingle{}}\NormalTok{,uNut,}\StringTok{\textquotesingle{}}\SpecialCharTok{\textbackslash{}n}\StringTok{\textquotesingle{}}\NormalTok{)}
\end{Highlighting}
\end{Shaded}

\begin{verbatim}
## Unique nut stations:  apacpnut apadbnut apaebnut apaegnut apaesnut apambnut apanhnut apapcnut aparvnut apascnut apawpnut
\end{verbatim}

\begin{Shaded}
\begin{Highlighting}[]
\FunctionTok{cat}\NormalTok{(}\StringTok{\textquotesingle{}Unique WQ  stations: \textquotesingle{}}\NormalTok{,uWQ,}\StringTok{\textquotesingle{}}\SpecialCharTok{\textbackslash{}n}\StringTok{\textquotesingle{}}\NormalTok{)}
\end{Highlighting}
\end{Shaded}

\begin{verbatim}
## Unique WQ  stations:  apabpwq apacpwq apadbwq apaebwq apaeswq apalmwq apapcwq
\end{verbatim}

\subsubsection{User Input Section}\label{user-input-section}

The following code assumes water quality (WQ) data, but can be edited
for nutrient or met data.

\begin{Shaded}
\begin{Highlighting}[]
\DocumentationTok{\#\#\#\#\#\#\#\#\#\#\#\#\#\#\#\#\#\#\#\#\#\#\#\#\#\#\#\#\#\#\#\#\#\#\#\#\#\#\#\#\#\#\#\#\#\#\#\#\#\#\#\#\#\#\#\#\#\#\#\#\#\#\#\#\#\#\#\#\#\#\#\#\#\#\#\#\#\#\#\#\#\#\#\#\#}
\CommentTok{\# Begin user input section}

\CommentTok{\# Load required libraries}
\FunctionTok{library}\NormalTok{(here)        }\CommentTok{\# Provides an easy way to construct file paths}
\end{Highlighting}
\end{Shaded}

\begin{verbatim}
## here() starts at /Users/Hulali/Documents/SWMPCourse/2025FallSWMPCourse
\end{verbatim}

\begin{Shaded}
\begin{Highlighting}[]
\CommentTok{\# The following two parameters are not necessarily known apriori; code above prints the unique stations}
\CommentTok{\# Select the nth station from the site}

\CommentTok{\#This will be used for plotting the data below}

\NormalTok{nSta }\OtherTok{=} \DecValTok{2}
\CommentTok{\# Select the nth variable}
\NormalTok{nCol }\OtherTok{=} \DecValTok{7} \CommentTok{\# 65 is turb, 37 is DO mean, 43 is depth}


\NormalTok{whichRows }\OtherTok{=} \FunctionTok{which}\NormalTok{(datWQ}\SpecialCharTok{$}\NormalTok{station}\SpecialCharTok{==}\NormalTok{uWQ[nSta])}
\CommentTok{\# End user input section}
\DocumentationTok{\#\#\#\#\#\#\#\#\#\#\#\#\#\#\#\#\#\#\#\#\#\#\#\#\#\#\#}
\end{Highlighting}
\end{Shaded}

\subsubsection{Plotting the data}\label{plotting-the-data}

\begin{Shaded}
\begin{Highlighting}[]
\CommentTok{\# Plot the original data}
\FunctionTok{par}\NormalTok{(}\AttributeTok{mfrow=}\FunctionTok{c}\NormalTok{(}\DecValTok{2}\NormalTok{,}\DecValTok{1}\NormalTok{),}\AttributeTok{lend=}\DecValTok{2}\NormalTok{,}\AttributeTok{mai =} \FunctionTok{c}\NormalTok{(}\FloatTok{0.25}\NormalTok{,}\FloatTok{0.75}\NormalTok{, }\FloatTok{0.08}\NormalTok{, }\FloatTok{0.05}\NormalTok{),}\AttributeTok{oma =} \FunctionTok{c}\NormalTok{(}\DecValTok{2}\NormalTok{,}\DecValTok{1}\NormalTok{,}\FloatTok{0.2}\NormalTok{,}\FloatTok{0.2}\NormalTok{), }\AttributeTok{cex =} \FloatTok{0.8}\NormalTok{)}
\NormalTok{myDS }\OtherTok{=} \FunctionTok{data.frame}\NormalTok{(}\AttributeTok{YearFrac=}\NormalTok{datWQ}\SpecialCharTok{$}\NormalTok{YearFrac[whichRows],}\AttributeTok{myData =}\NormalTok{ datWQ[whichRows,nCol])}
\FunctionTok{plot}\NormalTok{(myDS,}\AttributeTok{type=}\StringTok{\textquotesingle{}l\textquotesingle{}}\NormalTok{,}\AttributeTok{xlab =} \StringTok{\textquotesingle{}Year\textquotesingle{}}\NormalTok{,}\AttributeTok{ylab=}\FunctionTok{colnames}\NormalTok{(datWQ[nCol]))}

\CommentTok{\# Create a timeseries object}
\NormalTok{myTS }\OtherTok{\textless{}{-}} \FunctionTok{ts}\NormalTok{(myDS[,}\DecValTok{2}\NormalTok{], myDS[}\DecValTok{1}\NormalTok{,}\DecValTok{1}\NormalTok{], }\AttributeTok{frequency=}\DecValTok{12}\NormalTok{)}

\CommentTok{\# find and replace NAs with mean of the time series; other infill techniques could be used}
\NormalTok{iNA }\OtherTok{=} \FunctionTok{which}\NormalTok{(}\FunctionTok{is.na}\NormalTok{(myTS))}
\NormalTok{myTS[iNA] }\OtherTok{=} \FunctionTok{mean}\NormalTok{(myTS,}\AttributeTok{na.rm=}\ConstantTok{TRUE}\NormalTok{)}

\CommentTok{\# Plot the time series to compare with the original data}
\FunctionTok{plot}\NormalTok{(myTS)}
\end{Highlighting}
\end{Shaded}

\includegraphics{Timeseries_files/figure-latex/Plots-1.pdf}

\begin{Shaded}
\begin{Highlighting}[]
\CommentTok{\# Decompose timeseries into trend, seasonal, and random components}
\NormalTok{myTSdecomposed }\OtherTok{=} \FunctionTok{decompose}\NormalTok{(myTS)}
\CommentTok{\# Plot the decomposed timeseries}
\FunctionTok{plot}\NormalTok{(}\FunctionTok{decompose}\NormalTok{(myTS)) }
\end{Highlighting}
\end{Shaded}

\includegraphics{Timeseries_files/figure-latex/Plots-2.pdf}

\begin{Shaded}
\begin{Highlighting}[]
\CommentTok{\# Plot the autocorrelation function (ACF) for the trend component of the decomposed time series}
\CommentTok{\# Interpretation: The ACF plot for the trend shows how strongly the current value is correlated}
\CommentTok{\# with past values at various time lags. A slow decay of the ACF suggests a long{-}term dependency in the data.}
\FunctionTok{acf}\NormalTok{(}\FunctionTok{na.omit}\NormalTok{(myTSdecomposed}\SpecialCharTok{$}\NormalTok{trend), }\AttributeTok{ylab=}\StringTok{\textquotesingle{}ACF Trend\textquotesingle{}}\NormalTok{)}

\CommentTok{\# Plot the autocorrelation function for the seasonal component}
\CommentTok{\# Interpretation: The ACF plot for the seasonal component will often show a periodic pattern,}
\CommentTok{\# indicating that the data repeats at regular intervals (e.g., annual cycles). Peaks at regular lags}
\CommentTok{\# suggest strong seasonality.}
\FunctionTok{acf}\NormalTok{(}\FunctionTok{na.omit}\NormalTok{(myTSdecomposed}\SpecialCharTok{$}\NormalTok{seasonal), }\AttributeTok{ylab=}\StringTok{\textquotesingle{}ACF Seasonal\textquotesingle{}}\NormalTok{)}
\end{Highlighting}
\end{Shaded}

\includegraphics{Timeseries_files/figure-latex/Plots-3.pdf}

\begin{Shaded}
\begin{Highlighting}[]
\CommentTok{\# Plot the autocorrelation function for the random (residual) component}
\CommentTok{\# Interpretation: The ACF plot for the random component should ideally show no significant}
\CommentTok{\# correlation at any lag, as the residuals are expected to be random noise. Any significant correlations}
\CommentTok{\# might suggest some underlying pattern or structure still remaining in the residuals.}
\FunctionTok{acf}\NormalTok{(}\FunctionTok{na.omit}\NormalTok{(myTSdecomposed}\SpecialCharTok{$}\NormalTok{random), }\AttributeTok{ylab=}\StringTok{\textquotesingle{}ACF Random\textquotesingle{}}\NormalTok{)}
\end{Highlighting}
\end{Shaded}

\includegraphics{Timeseries_files/figure-latex/Plots-4.pdf}

\end{document}
